\chapter{Prologue}

One of the greatest challenges in fisheries management is to understand the population variation of target species, a process mainly related to its recruitment \citep{Cush1971,Lask1981,MaunThor2019}. In many pelagic fish species, early life stages are critical to survive and to recruit, showing a density- and size-dependent mortality \citep{StigRoge2019}, furthermore growing rapidly increases the chances of a larval stage individual to survive and recruit \citep{OsseBoog1997,Soga1997,MeekVigl2006}, then, favorable environmental conditions as well as with moderate turbulence, food availability and in synchrony with a spawning process, favors a rapid growth of larvae and therefore a higher recruitment \citep{CuryRoy1989}.\\

We provide new insights into the understanding of Peruvian anchovy (\textit{Engraulis ringens}) recruitment, understood as a dynamic and complex process over different life cycle stages \citep{DuffBail2005}, but we emphasized on early life stages and how ocean currents, temperature and food availability impact on the growth and subsequent recruitment.\\

With this general objective in mind, this thesis is organized in 4 chapters. 1) to provide an introduction with the theoretical framework of the state of the art of the ecosystem and the species; 2) to explore the coastal retention process using a lagrangian modeling tool, which gives us an understanding of the potential presence of eggs and larvae in the continental shelf off Peru, catalogued as favorable to support high biomasses of various species; 3) to evaluate the combined effect of temperature and food availability on growth and recruitment of \textit{E. ringens}; and 4) to develop a full life cycle model of \textit{E. ringens}. 