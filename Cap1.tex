\chapter{Introduction and state of the art}\label{Chap1}

\section{Upwelling systems}\label{Chap1UpweSyst}

Oceanography as a science dates back to the late 19th century \citep{Wust1964,Mill2012,LlopCowe2014} and studies the oceans and everything related to them, from physical processes to biological ones, that make it a multidisciplinary science. One of the most important physical processes is upwelling whose study interest dates back to the beginning of the 20th century \citep{Ogil1912,Murp1920}. Upwelling zones are sites where a fertilization process regularly occurs due to the upward movement of water mass parcels in the water column and to be efficient this process should maintain long enough, at least a few days, in order to upwell deep nutrient-rich water parcels to the surface over a vertical distance of sometimes 100 m or more \citep{Marg1978}. These systems are critical for marine ecosystems and have significant implications for climate and weather patterns since climate should be particularly sensitive to hydrological processes, especially in the tropics \citep{Webs1994}. There are several key upwelling systems around the world, and they typically occur along the western coastlines of continents. The most important form of upwelling is wind-driven upwelling (Fig. \ref{Chap1CoastalUpwelling}), although there are other forms such as Ekman pumping or equatorial upwelling that will not be discussed here.\\

\begin{figure}[ht]
	\centering
	\includegraphics[width=1.0\textwidth]{figures/Chap1CoastalUpwelling.png}
	\caption{Conceptual model of wind-driven upwelling process in the northern hemisphere. It should be noted that in the southern hemisphere, the wind direction is reversed but generates the same upwelling effect.}
	\label{Chap1CoastalUpwelling}
\end{figure}

Since the seawater is a practically incompressible fluid, when there is a vertical upward flow then an equivalent volume of water per unit of time must be displaced in the horizontal direction, when upwelling occurs along the coast with landmass as a boundary, the horizontal displacement is necessarily directed offshore \citep{KampCap2}. Wind-driven coastal upwelling is an efficient way of transporting nutrients from deep to shallow (euphotic) zones in the water column \citep{MessChav2015,MessChav2017} but it is also a source of environmental stress due to widely changing conditions in temperature \citep{CastWang2014}, dissolved oxygen \citep{Scul2010}, and salinity \citep{XuanHuan2012}.\\

The consequence of the upwelling process is the fertilization of the photic zone (Fig. \ref{Chap1UpwellingFertilization}), with waters coming from the deep supplying nutrients such as nitrogen (\textit{N}), phosphorous (\textit{P}), silica (\textit{Si}) and iron (\textit{Fe}, an element that is often limiting) to the surface \citep{GeidLaro1994,Behr1996,BehrKolb1999,Hutc2002,MoorMils2013,BrisMohr2017}, although nutrient supply may also come from other sources such as river discharge \citep{Shar2017} or from the atmosphere \citep{Powe2015}.\\

\begin{figure}[ht]
	\includegraphics[width=1.0\textwidth]{figures/Chap1UpwellingFertilization.jpg}
	\centering
	\caption{Scheme of fertilization sources to the ocean as well as upwelling, riverine discharges and atmospheric deposition. The figure was taken from \cite{BrisMohr2017}.}
	\centering
	\label{Chap1UpwellingFertilization}
\end{figure}

The upwelling process has important ecological consequences only if parcels of water transported from the bottom to the photic zone are rich in nutrients. This occurs in what are called upwelling systems, of which the area off Peru is one of the most important \citep{ChavBert2008}, although  a global trend of decreasing primary productivity has been shown in the last decades \citep{Dema2009,Roxy2016} and in climate change scenarios \citep{Blanc2012,Kulk2020}, which could potentially jeopardize the amount of primary production required to sustain global fisheries \citep{PaulChri1995}, especially the Peruvian anchovy fishery, which has had an historic collapse in the past \citep{AriaNiqu2011,Aria2012}, and could play an important role in the food security of the Peruvian people \citep{MajlDela2017}. We also know that the environmental conditions in which the Peruvian anchovy develops are favorable, although they also present great variability in the northern zone of the Humboldt current.\\

Understanding upwelling systems is essential for managing fisheries, studying climate impacts, and preserving the health of marine ecosystems.\\

\clearpage

\section{The northern Humboldt Current system}\label{Chap1NHCS}

The northern Humboldt Current system (NHCS) is one of the most productive systems in the world (Fig. \ref{Chap1UpwellingSystems}), mostly located off the Peruvian coast (70\textdegree W – 90\textdegree W; 0\textdegree S – 20\textdegree S) and is part of the broader Peru-Chile upwelling system \citep{GradChai2018}. NHCS is considered as a highly productive marine ecosystem with a productivity of $>300 g C/cm^{2}/yr$ \citep{KampCap5}, largely due to the regular presence of upwelling-favorable winds throughout the year from 5\textdegree S to 26\textdegree S, a feature that is rather seasonal beyond 26\textdegree S \citep{Belm2014}. There is however seasonal and interannual variability in upwelling intensity, with intensification in the later part of the 20th century (1960 – 2001) \citep{Nara2010}.\\

\begin{figure}[ht]
	\includegraphics[width=1.0\textwidth]{figures/Chap1UpwellingSystems.png}
	\centering
	\caption{Scheme of the four most important upwelling systems represented by chlorophyll concentration measured by satellite images. The figure was taken from \cite{MessChav2015}.}
	\label{Chap1UpwellingSystems}
\end{figure}

In the NHCS, water masses from different origins with specific temperature and salinity ranges converge \citep{SilvRoja2009,MontCola2010,ChaiDomi2013}. The following water masses are currently distinguished off Peru: Tropical Surface Water (TSW); Equatorial Surface Water (ESW); Subtropical Surface Water (STSW); Equatorial Subsurface Water (ESSW); Eastern South Pacific Intermediate Water (ESPIW); Antarctic Intermediate Water (AAIW) \citep{GradChai2018}. Details of temperature and salinity range of these water masses are reported in Table \ref{TabWaterMasses} and Fig. \ref{Chap1WaterMassesNHCS}. These water masses are transported by several surface [Ecuador-Peru Coastal Current (EPCC); Peru Coastal Current (PCC); South Equatorial Current (SEC); Peru Oceanic Current (POC)] and subsurface currents [Equatorial Undercurrent (EUC); Peru-Chile Undercurrent (PCUC)].\\

\begin{table}
\centering
\begin{tabular}{c|c|c}
\hline
\textbf{Water mass}&\textbf{Temperature range (\textdegree C)}&\textbf{Salinity range (S)}\\
\hline
TSW   & 23.5 - 24.5 & 33.5 – 34.4        \\
ESW   & 20.0 – 24.0 & 34.6 – 35.0        \\
STSW  & 19.0 – 23.5 & \textgreater{}35.4 \\
ESSW  & 8.0 – 14.0  & 34.6 – 35.0        \\
ESPIW & 12.0 – 14.0 & 34.8               \\
AAIW  & 4.0 – 7.0   & 34.5 – 34.6             
\end{tabular}
\caption{Temperature (T, in \textdegree C) and salinity (S) ranges of the main water-masses in the NHCS. Tropical Surface Water (TSW); Equatorial Surface Water (ESW); Subtropical Surface Water (STSW); Equatorial Subsurface Water (ESSW); Eastern South Pacific Intermediate Water (ESPIW); Antarctic Intermediate Water (AAIW). Table modified from \cite{GradChai2018}.}
\label{TabWaterMasses}
\end{table}

\begin{figure}[ht]
	\includegraphics[width=1.0\textwidth]{figures/Chap1WaterMassesNHCS.png}
	\centering
	\caption{Main water-masses of the NHCS. The color corresponds to the salinity range. The surface (thick solid lines) and subsurface (thick dashed lines) currents: Ecuador-Peru Coastal Current (EPCC); Peru Coastal Current (PCC); South Equatorial Current (SEC); Peru Oceanic Current (POC); Equatorial Undercurrent (EUC); Peru-Chile Undercurrent (PCUC). Figure taken from \cite{GradChai2018}.}
	\label{Chap1WaterMassesNHCS}
\end{figure}

The mean circulation shows both poleward and equatorward flows (Fig. \ref{Chap1MeanCirculationNHCS}). The upper layer ($\sim$25 m) close to the coast shows a relatively strong poleward flow ($\sim$20 – 30 $cm/s^{-1}$) associated with the EPCC transporting relatively warm and fresh ESW. At 5\textdegree S, the EPCC separates into two branches, one feeding the westward SEC and the other one continuing southward weakening until $\sim$8\textdegree S – 9\textdegree S.\\

Further south there is equatorward flow associated with the PCC and the POC transporting cold waters. The middle layer (100 – 200 m) is predominantly a poleward flow influenced by the PCUC that extends along the entire coast, transporting ESSW with a mean velocity that can locally reach 20 $cm/s^{-1}$. In the deeper layer (500 m), the average circulation is mainly equatorward transporting relatively fresh and cold AAIW \citep{ChaiDomi2013,PietTest2013}.\\

The mean temperature in the coastal zone off Peru is 17.6\textdegree C \citep{Mont2003} but temperature shows strong seasonal and interannual variability, particularly due to El Niño events that can generate positive anomalies of more than 6\textdegree C  \citep{Sanc2000,Cai2014,Cai2017,Cai2018,Freu2019}, and other extreme events referred to as "coastal Niño" \citep{Eche2018,Garr2018,Hu2019,Rodr2019,TakaMart2019}. Dissolved oxygen also shows a high spatiotemporal variability, which may at time limit the generally well oxygenated layers located near the surface \citep{EspiEche2017,EspiEche2019}.\\

Thus, water masses intermingle along the coastline following the currents and constitute the biotope of one of the most productive ecosystems in the world. The small pelagic fish reproduce by dispersing their eggs in the surface layers, and are therefore particularly exposed to the fluctuations of this biotope which will determine in large part their chances of survival. In the following section we will describe the main species of small pelagic fish that inhabit this ecosystem, the Peruvian anchovy.\\

\begin{figure}[ht]
	\includegraphics[width=1.0\textwidth]{figures/Chap1MeanCirculationNHCS.png}
	\centering
	\caption{Mean circulation at 25 m, between 100 m and 200 m, and 500 m depth. Red and blue arrows represent the equatorward and poleward flows, respectively. Green arrows indicate the presence of mesoscale cyclonic ($\sim$14\textdegree S) and anticyclonic ($\sim$17\textdegree S) eddy-like features. Figure taken from \cite{ChaiDomi2013}.}
	\label{Chap1MeanCirculationNHCS}
\end{figure}

\clearpage

\section{The Peruvian anchovy (\textit{Engraulis ringens}, Jenyns 1842)}\label{Chap1PeruAnch}

Fig. \ref{Chap1Engraulis_ringens} shows a schematic representation of \textit{E. ringens}, a small pelagic species with a slightly compressed elongated body, long head, prolonged snout and very large eyes. In the dorsal part its color is dark blue, while in the ventral part it is silver \citep{Whit1988}.\\

Taxonomically, according to Integrated Taxonomic Information System, \textit{E. ringens} is located in the evolutionary tree as follows:\\

\begin{itemize}
  \centering
  \item Kingdom: Animalia
  \item Phylum: Chordata
  \item Subphylum: Vertebrata
  \item Infraphylum: Gnathostomata
  \item Superclass: Actinopterygii
  \item Class: Teleostei
  \item Order: Clupeiformes
  \item Family: Engraulidae
  \item Genus: Engraulis
  \item Species: \textit{Engraulis ringens}, Jenyns, 1842
  \item Common name: Peruvian anchovy
\end{itemize}

\begin{figure}[ht]
	\includegraphics[width=1.0\textwidth]{figures/Chap1Engraulis_ringens.jpg}
	\centering
	\caption{Schematic representation of an adult Peruvian anchovy (\textit{Engraulis ringens}).}
	\label{Chap1Engraulis_ringens}
\end{figure}

\textit{E. ringens} can reach a maximum size of 20 cm and has a life span of 4 years. Its first sexual maturity may be at a size of 12 cm \citep{GutiSwar2007,MarzShin2009}. This species performs external fertilization by formation of spawning aggregations, which warrants synchrony in the timing of gamete production and optimizes fertilization rates. Females release the eggs into the environment so that males can fertilize them without the need for copulation \citep{Gani2014}. In unfavorable environmental conditions, a female is capable of reabsorbing part of the ovigerous mass to obtain energy, a phenomenon known as atresia \citep{PereRoqu2008,EspiVera2009,ClarCast2012,BuitPere2018}. This species has two important spawning peaks off Peru, summer and winter, but a significant presence of eggs and larvae is reported throughout the year \citep{MarzShin2009}. Anchovy in Peru has a mainly zooplanktivorous diet, with ontogenetic and spatiotemporal variability in prey preferences, with an increase in the contribution of euphausiids to the diet as \textit{E. ringens} increases in size up to 85 \% of their diet as adults (18 – 20 cm), in contrast, their consumption of calanoid copepods decreases with size, while diatoms make up a minimal part of their diet throughout their life span \citep{EspiBert2008,EspiBert2014}.\\

\textit{E. ringens}’s early life description \citep{RiouOfel2021}, currently distinguishes 4 stages: stage 1 from hatching to 2 days post-hatch (dph) there are transparent yolk-sac larvae with closed mouth and eyes and non-pigmented body. Stage 1 ends with the opening of the mouth, pigmentation of the eyes and remnants of the yolk sac (at length ranging from ~ 2 mm to 4 mm depending on temperature); stage 2 from the complete absorption of the yolk sac at 3 dph until larvae are between 12 and 19 dph with length ranging from 8.15 mm to 12.97 mm; stage 3 included larvae between 19 and 26 dph and corresponding length from 10.89 to 15.03 mm; stage 4, from 33 dph, larvae showed melanophores developed on the head and body, with visible gills. The first sign of schooling (continuous swimming as a group and clear evidence of active swimming behaviour, \cite{Shaw1962}) also occurred at this stage (31 dph).\\

After the larval stage, anchovy juveniles already have the definitive adult form and from 12 cm of length, they became available for industrial fishery \citep{MarzShin2009}. This industrial fishery is oriented almost entirely (98 \%) to indirect human consumption (feed fish) and only 2 \% is intended for direct human consumption (food fish) in canned, cured, frozen or fresh presentations \citep{FreoSuei2014}.\\

\textit{E. ringens} fishery management is based on scientific monitoring of adult population indicators determined by acoustic surveys \citep{GutiSwar2007} or by the daily egg production method \citep{Ayon2000}, then allocating fishing quotas to each registered vessel and catching monitoring by technical research personnel on board these vessels \citep{KroeSanc2019}. Thus, the Peruvian anchovy has become one of the most abundant and at the same time the most studied fishery, but understanding the impact of environmental conditions on the early-life stages of anchovy and further population dynamics remains challenging.\\

\begin{figure}[ht]
	\includegraphics[width=1.0\textwidth]{figures/Chap1VertebraeAbundances.png}
	\centering
	\caption{Fish vertebrae abundances during the last interglacial and Holocene. Figure taken from \cite{Salv2022}.}
	\label{Chap1VertebraeAbundances}
\end{figure}

However, this abundance has not been constant at geological scales as shown by palaeoceanographic records, which have evidenced that \textit{E. ringens} has populated the NHCS for thousands of years under a naturally varying climate with high variability during the last 25 ky \citep{Salv2018,Salv2019,Salv2022}, playing an important role transferring energy from lower to higher levels of the food web \citep{ChecAsch2017}. Fig. \ref{Chap1VertebraeAbundances} show we are currently in the Holocene period, characterized by a great abundance of the Peruvian anchovy biomass, however, during the last interglacial period, with warm and oxygen-poor environmental conditions, the presence of \textit{E. ringens} was considerably lower, and that conditions favored the flourishing of other mesopelagic fishes \citep{Salv2022}, environmental conditions that could be repeated in a climate change scenario \citep{EcheGeva2020}.\\

Thus, environmental variability in a complex system such as the NHCS and its impact on \textit{E. ringens} abundance has been extensively studied under different approaches, one of them being modeling.\\

\clearpage

\section{Modeling studies on \textit{E. ringens}}\label{Chap1ModeAnch}

The inherent impossibility of monitoring the entire ecosystem at high frequency and resolution is a major limitation of ichthyoplankton in situ observations. This is where models, become relevant, allowing us to integrate multiple spatio-temporal scales and to testing hypotheses on ichthyoplankton dynamics in response to the different  environmental forcing’s, including forecasting scenarios \citep{GearDode2020}.\\

Despite many modeling studies conducted in the past \citep{LettPenv2007,BrocLett2008,GutiRami2008,OlivPena2011,XuChai2013} that have tried to explain a relationship between environmental variability and the success of anchovy recruitment, much remains to be understood due to the high plasticity of this species and its adaptation capacity to environmental changes \citep{EspiBert2008,EspiBert2014,CanaAdas2018,PlazCern2018}.\\

Early modeling studies for the understanding of NHCS dynamics were done by hydrodynamic modeling reproducing general ocean circulation patterns \citep{PenvEche2005,ColaMcwi2012}, its interannual variability \citep{ColaCape2008,EspiEche2017}, its potential changes under future climate scenarios \citep{OerdCola2015,EcheGeva2020}. These previous works provide us a strong physical basis for ecological studies in the NHCS.\\

From a realistic representation of the physical environment, researchers have tested \cite{Baku1998} triad hypothesis for small pelagic fish recruitment and early life stages survival by quantification of enrichment, concentration and retention processes in NHCS \citep{LettPenv2007} and complementarily also in the Benguela upwelling ecosystem \citep{LettRoy2006}, in both cases, the coastal zone emerged as the most favorable by analyzing the spatial distribution and seasonal variability of these indices. Further modelling studies showed that \textit{E. ringens} larval retention patterns displayed strong seasonal variability possibly modulated by vertical distribution of individuals in regards to the vertical current structure, resulting in a maximal coastal retention close to the surface in winter and in deeper layers in summer. Also, a partial match between dates and locations of enhanced retention and observed egg concentration patterns was found \citep{BrocLett2008}.\\

In the four main eastern boundary upwelling systems (EBUS), the reproductive strategy of small pelagic fish species (SPF), dominated by anchovies and sardines, is a trade-off between larval retention and larval food availability. In the NHCS, the winter spawning benefit both from high zooplankton and shelf retention rates, which may explain the particularly  large SPF populations \citep{BrocCola2009,BrocLett2011}. Climate change might negatively impact \textit{E. ringens} recruitment due to a reduction in the size of the habitat due to a reduction in ecosystem productivity not compensated by increased retention rates \citep{BrocEche2013}. The effects of temperature and food availability on larval growth could modulate this result; as it was suggested that ENSO events may negatively impact \textit{E. ringens}  recruitment by reducing the larval growth speed due to changes in food availability \citep{XuChai2013,XuRose2015}. The seasonal variability in larval growth speed was not explicitly taken into account in previous studies, a gap that was addressed in the present work.\\

In order to investigate the impact of environmental variability on early life stages of small pelagic fishes, we developed Ichthyop-DEB, an individual-based model \citep{LettVerl2008} including larval retention processes and a Dynamic Energy Budget \citep{Kooi2009} bioenergetic module for larval growth. Using this tool, we assessed the effect of hydrodynamic simulations horizontal resolution on simulated larval retention patterns in in chapter 2, then, we studied the impact of the biological processes on simulated larval growth and recruitment and the effect of the upper thermal limit in chapter 3, for which lab experiments are lacking, finally, we presented a complete life cycle growth model of the Peruvian anchovy with potential uses for the study of juvenile and adult behaviour.\\

\clearpage