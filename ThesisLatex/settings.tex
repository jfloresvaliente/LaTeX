% Paquetes y configuraciones para Sorbone Universite

% Define the margin dimensions based on paper format
\newcommand{\setmargins}{
	\ifdim\paperwidth=\pdfpagewidth
	% A4 format
	\usepackage[margin=40mm]{geometry}
	\else
	% B5 format
	\usepackage[margin=25mm]{geometry}
	\fi
}
\setmargins

\usepackage[british]{babel}
\usepackage{titling,graphicx,verbatim,pdfpages,enumerate,multicol}
\usepackage{subcaption}
\usepackage[utf8]{inputenc}

% Line spacing
\renewcommand{\baselinestretch}{1.06}

\frenchspacing

\newcommand{\University}{Sorbonne Universit\'e, Paris-France}
\newcommand{\Thesisdesc}{Dissertation for the degree of PhD}

\usepackage{fancyhdr}
\pagestyle{fancy}
\fancypagestyle{plain}{% first page of chapter
	\fancyhf{} % comment to keep the normal header
	\renewcommand{\headrulewidth}{0pt} % comment to keep rule
	\fancyfoot{} % comment to put page number at bottom
}
% It is necessary to configure the width of the 'header' on each page.
\setlength{\headheight}{41pt}
\renewcommand{\chaptermark}[1]{\markboth{\small #1}{}}
\renewcommand{\sectionmark}[1]{\markright{\small \thesection\ #1}{}} 
\renewcommand{\headrulewidth}{0.6pt}
\fancyhead[LE,RO]{\thepage}
\fancyhead[LO]{\rightmark}
\fancyhead[RE]{\leftmark}
\fancyfoot{} % no footer

\usepackage{microtype,booktabs}
\usepackage{mathptmx,amsmath}
\usepackage{natbib}
\usepackage{enumitem} 

% To alter the color of cross-references and bibliographic citations.
\usepackage[colorlinks,urlcolor=green,citecolor=red,linkcolor=blue]{hyperref}
\usepackage{cleveref} % this package must be loaded after 'hyperref'

% Change the title: 'Fig.' instead of 'Figure'
\usepackage[figurename=Fig.]{caption}

\Crefname{figure}{Figure}{Figures}
\Crefname{equation}{Equation}{Equations}
\Crefname{table}{Table}{Tables}
\crefname{figure}{Fig.}{Figs.}
\crefname{equation}{Eq.}{Eqs.}
\crefname{table}{Table}{Tables}

% The directory in which the figures are stored
\graphicspath{{figures/}} 

\usepackage[font={small,it}]{caption}
\usepackage[normalem]{ulem}
\useunder{\uline}{\ul}{}

% Generate vertical environments to rotate a figure or table
\usepackage{lscape}

% Enumerate lines of text
\usepackage{lineno}

% Set the levels in the list of contents
\setcounter{tocdepth}{3}
\setcounter{secnumdepth}{3}

% Control cell dimensions in tables
\usepackage{array}

% Put 2, or more lines in a cell
\usepackage{makecell}

% Add glossary
\usepackage[style=indexgroup, toc,section=chapter]{glossaries}
\makenoidxglossaries % use TeX to sort
\loadglsentries{glossaries}

% Include 'References' in the table of contents.
\usepackage[nottoc]{tocbibind}

% Adjust the table
\usepackage{adjustbox}

% Add notes within the table
\usepackage{enumitem}
\usepackage{nicematrix}

% Using mathematical symbols
\usepackage{amssymb}

% Put the title of the tables in the upper part of the table
\usepackage{floatrow}
\floatsetup[table]{capposition=above}

% Coloring tables (one line yes, one line no)
\usepackage{colortbl}
\definecolor{gris}{RGB}{242,242,242}
\usepackage[table]{xcolor}

% Para agregar texto en formato de código
\usepackage{listings}

%\definecolor{codegreen}{rbg}{0,0.6,0}
%\definecolor{codegray}{rbg}{0.5,0..5,0.5}
%\definecolor{codepurple}{rbg}{0.58,0,0.82}
%\definecolor{backcolor}{rbg}{0.95,0.95,0.92}

\lstdefinestyle{mystyle}{
    %backgroundcolor=\color{black},
    %commentstyle=\color{green},
    %keywordstyle=\color{magenta},
    %numberstyle=\tiny\color{gray},
    %stringstyle=\color{purple},
    basicstyle=\ttfamily\footnotesize,
    breakatwhitespace=false,
    breaklines=true,
    captionpos=t,
    keepspaces=true,
    %numbers=left,
    showspaces=false,
    showstringspaces=false,
    showtabs=false,
    tabsize=2
}

% agregar apendice
\usepackage{appendix}

% agregar pdf
\usepackage{pdfpages}