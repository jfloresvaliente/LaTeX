% Paquetes y configuraciones para Sorbone Universite

% Define the margin dimensions based on paper format
\newcommand{\setmargins}{
	\ifdim\paperwidth=\pdfpagewidth
	% A4 format
	\usepackage[margin=40mm]{geometry}
	\else
	% B5 format
	\usepackage[margin=25mm]{geometry}
	\fi
}
\setmargins

\usepackage[british]{babel}
\usepackage{titling,graphicx,verbatim,pdfpages,enumerate,multicol}
\usepackage{subcaption}
\usepackage[utf8]{inputenc}

% distancia entre lineas
\renewcommand{\baselinestretch}{1.06}

\frenchspacing

\newcommand{\University}{Sorbonne Universit\'e, Paris-France}
\newcommand{\Thesisdesc}{Dissertation for the degree of PhD}

\usepackage{fancyhdr}
\pagestyle{fancy}
\fancypagestyle{plain}{% first page of chapter
	\fancyhf{} % comment to keep the normal header
	\renewcommand{\headrulewidth}{0pt} % comment to keep rule
	\fancyfoot{} % comment to put page number at bottom
}
% Configurar el ancho del "header" de cada pagina
\setlength{\headheight}{41pt}
\renewcommand{\chaptermark}[1]{\markboth{\small #1}{}}
\renewcommand{\sectionmark}[1]{\markright{\small \thesection\ #1}{}} 
\renewcommand{\headrulewidth}{0.6pt}
\fancyhead[LE,RO]{\thepage}
\fancyhead[LO]{\rightmark}
\fancyhead[RE]{\leftmark}
\fancyfoot{} % no footer

\usepackage{microtype,booktabs}
\usepackage{mathptmx,amsmath}
\usepackage{natbib}
\usepackage{enumitem} 

% Para cambiar el color de las referencias cruzadas y citas bibliograficas
\usepackage[colorlinks,urlcolor=green,citecolor=red,linkcolor=blue]{hyperref}
\usepackage{cleveref} % este paquete debe cargarse despues de "hyperref"

% Cambiar el titulo en la figuras Fig. en lugar de Figure
\usepackage[figurename=Fig.]{caption}

\Crefname{figure}{Figure}{Figures}
\Crefname{equation}{Equation}{Equations}
\Crefname{table}{Table}{Tables}
\crefname{figure}{Fig.}{Figs.}
\crefname{equation}{Eq.}{Eqs.}
\crefname{table}{Table}{Tables}

% directorio donde se ubican las figuras
\graphicspath{{figures/}} 

\usepackage[font={small,it}]{caption}
\usepackage[normalem]{ulem}
\useunder{\uline}{\ul}{}

% Generar ambientes verticales y voltear una figura o tabla
\usepackage{lscape}

% Enumerar lineas de texto
\usepackage{lineno}

% Establecer los niveles en la lista de contenidos
\setcounter{tocdepth}{2}

% Controlar las dimensiones de celdas en tablas
\usepackage{array}

% Agregar glosario
\usepackage[style=indexgroup, toc,section=chapter]{glossaries}
\makenoidxglossaries % use TeX to sort
\loadglsentries{glossaries}

% Incluir la bibliografia en la tabla de contenidos
\usepackage[nottoc]{tocbibind}