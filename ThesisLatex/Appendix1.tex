\chapter{Supplementary material}\label{App1}

\clearpage
\section*{DEB parameter estimation codes}

\lstinputlisting[language=matlab,style=mystyle, caption = $run\_Engraulis\_ringens.m$]{code/run_Engraulis_ringens.m}
\lstinputlisting[language=matlab,style=mystyle, caption = $mydata\_Engraulis\_ringens.m$]{code/mydata_Engraulis_ringens.m}
\lstinputlisting[language=matlab,style=mystyle, caption = $pars\_init\_Engraulis\_ringens.m$]{code/pars_init_Engraulis_ringens.m}
\lstinputlisting[language=matlab,style=mystyle, caption = $predict\_Engraulis\_ringens.m$]{code/predict_Engraulis_ringens.m}

\clearpage
\section*{Related Scientific Publications}

\includepdf[pages=-]{pub/Flores-Valiente2023.pdf}


%\section{Accelerated $DEB_{abj}$ Equations in Ichthyop-DEB model}\label{DEBabjEqn}

%Jorge Flores-Valiente et al.\\

%This section describes the complete life cycle \acrshort{deb} model for \gls{ringens}, with a focus on the egg and larval periods. We implemented these equations in the Lagrangian tool routines of \gls{ich} \citep{LettVerl2008} to develop \gls{ich-deb}.\\

%Table \ref{param_compar} presents a comparative list of parameters used in both bioenergetic models (\textit{\gls{encrasicolus}} and \textit{\gls{ringens}}) at a reference temperature $T_{1} = 293 K$  ($=20$\textdegree $C$).\\

%\subsection*{Forcing variables}
%\hfill \\

%$T$ Temperature (K) is the water temperature surrounding an individual (modeled by CROCO-PISCES)\\

%$X$ Food density averaged Mesozooplankton field ($\left[ \mu mol CL^{-1} \right]$) over the water column (modeled by CROCO-PISCES)\\

%\subsection*{State variables – Initial conditions}

%The age of the individual is set at zero on the day of spawning.\\

%\begin{tabular}{|c|c|c|c|}
%\hline 
%Symbol  & Value      & Units  & Definition      \\ 
%\hline 
%$E$     & $0.99$     & $J$    & Initial Reserve \\ 
%$V$     & $0.000001$ & $mm^3$ & Initial Structure\\
%$E_{H}$ & $0$ 		  & $J$    & Cumulated energy invested into development\\
%$E_{R}$ & $0$         & $J$   & Reproduction buffer\\
%\hline 
%\end{tabular}

%\subsection*{Scaled functional response}
%\hfill \\

%\textbf{if} $(E_{H} < E_{H}^b)$\\

%$f = 0$				\hfill No feeding.\\

%\textbf{else}\\

%$f= \frac{X}{X+K}$	\hfill Feeding.\\

%\subsection*{Temperature correction}
%\hfill \\

%Each rate parameter is corrected for temperature according to the following equation \citep{Kooi2009}.\\

%$
	%C_{T} = exp\left ( \frac{T_{A}}{T_{1}} - \frac{T_{A}}{T} \right )
	%\left ( \frac
		%		{1+exp\left ( \frac{T_{AL}}{T_{1}} - \frac{T_{AL}}{T_{L}} \right )
		%		  +exp\left ( \frac{T_{AH}}{T_{H}} - \frac{T_{AH}}{T_{1}} \right )}
		%		{1+exp\left ( \frac{T_{AL}}{T} - \frac{T_{AL}}{T_{L}} \right )
			%	  +exp\left ( \frac{T_{AH}}{T_{H}} - \frac{T_{AH}}{T} \right )}
	%\right )
%$\\

%With $T_{1}$ the reference temperature (at which flux parameters were estimated), $T_{A}$ is the Arrhenius temperature and $T_{AH}$, $T_{AH}$, $T_{L}$, $T_{H}$ are constants used to define a curved shape of the temperature correction according to temperature.\\

%$\left \{ \dot{p}_\mathrm{Am} \right \}_{T} = C_{T} \left \{ \dot{p}_\mathrm{Am} \right \}_{T_{1}}$\\

%$\left [ \dot{p}_{M} \right ]_{T} = C_{T} \left [ \dot{p}_{M} \right ]_{T_{1}}$\\


%Temperature correction now affects two additional parameters.\\

%$\dot{v}_{T} = C_{T} \dot{v}_{T_{1}}$\\

%$\dot{K}_{jT} = C_{T} \dot{K}_{jT_{1}}$\\

%\subsection*{Metabolic acceleration}
%\hfill \\

%Only two parameters are impacted by growth acceleration.\\

%\textbf{if}	$E_{H} < E_{H}^b$\\

%$S_{M} = 1$ \hfill No acceleration.\\

%\textbf{else if} $E_{H}^b < E_{H} < E_{H}^j$\\

%$S_{M} = \frac{L}{L_{b}}$ \hfill Acceleration.\\

%\textbf{else}\\

%$S_{M} = \frac{L_{j}}{L_{b}}$ \hfill Constant growth.\\

%The $S_{M}$ parameter only accelerates growth from birth to metamorphosis.\\

%$\left \{ \dot{p}_\mathrm{Am} \right \}_{S_{M}} = S_{M} \left \{ \dot{p}_\mathrm{Am} \right \}_{T}$\\

%$\dot{v}_{S_{M}} = S_{M} \dot{K}_{jT}$\\

%\subsection*{Fluxes ($Jd^{-1}$)}
%\hfill \\

%\textbf{if}	$E_{H} < E_{H}^b$\\

%$\dot{p}_\mathrm{A} = 0$ \hfill No assimilation.\\

%\textbf{else}\\

%$\dot{p}_\mathrm{A} = f \left \{ \dot{p}_\mathrm{Am} \right \}_{S_{M}} V^{2/3}$ \hfill Assimilation.\\

%\hfill \\

%$\dot{p}_\mathrm{M} = \left [ \dot{p}_{M} \right ]_{T} V$ \hfill Volume-related somatic maintenance.\\

%$\dot{p}_{C} = \frac{E}{V}*\frac{\left [ E_{G} \right ]\dot{v}_{S_{M}}V^{2/3}+\left [ \dot{p}_{M} \right ]_{T}}{\kappa\frac{E}{V}+\left [ E_{G} \right ]}$ \hfill Mobilization of energy.\\

%$\dot{p}_{J} = \dot{K}_{jT} E_{H}$ \hfill Maturity maintenance.\\

%\subsection*{Differential equations}
%\hfill \\

%$\frac{dE}{dt} = \dot{p}_\mathrm{A} - \dot{p}_{C}$ \hfill Reserve dynamics.\\

%$\frac{dV}{dt} = \frac{\kappa \dot{p}_{C} - \dot{p}_\mathrm{M}}{\left[E_{G} \right]}$\\

%\textbf{if} $E_{H} < E_{H}^p$\\

%$\frac{dE_{H}}{dt} = (1 - \kappa) \dot{p}_{C} - \dot{p}_{J}$\\

%$\frac{dE_{R}}{dt} = 0$\\

%\textbf{else}\\

%$\frac{dE_{H}}{dt} = 0$\\

%$\frac{dE_{R}}{dt} = (1 - \kappa) \dot{p}_{C} - \dot{p}_{J}$\\

%\subsection*{Integration}
%\hfill \\

%$E\left ( t + \Delta t \right ) = E\left ( t \right ) + \frac{dE}{dt}\Delta t$ \\

%$V\left ( t + \Delta t \right ) = V\left ( t \right ) + \frac{dV}{dt}\Delta t$ \\

%$E_{H}\left ( t + \Delta t \right ) = E_{H}\left ( t \right ) + \frac{dE_{H}}{dt}\Delta t$ \\

%$E_{R}\left ( t + \Delta t \right ) = E_{R}\left ( t \right ) + \frac{dE_{R}}{dt}\Delta t$ \\

%With $\Delta t = 0.083 d \left (=2hours\right )$

%\subsection*{Observable variables}
%\hfill \\

%$L_{w} = \frac{V^{1/3}}{\delta_{M}}$ \hfill Physical length ($cm$).\\

%where $L_{w}$ is the standard length $SL$ ($cm$), $V$ the structural volume ($cm^3$) and $\delta_{M}$ a shape coefficient.\\

%We assume that the larva does not change in shape till it reaches our length criteria of $2cm(SL)$ and that there is a constant proportionality ($\delta_{M}$) between structural volume and length.\\
