\chapter{Prologue}

Motivated to help understand one of the greatest challenges in fisheries management, I mean to comprehend the fluctuations in the population of the target species at various spatial and temporal scales, under the premise that this variation is a process highly linked to their recruitment \citep{Cush1971,Lask1981,MaunThor2019}, a concept whose meaning will depend rather on each author and the use he/she wants to give it, is that I started this research path. A review of past decades reveals that numerous efforts have been made to elucidate this phenomenon, applying different methodologies, often limited by the technology of the moment and also by the enormous complexity of monitoring a fish population in its natural environment.\\

It is intuitive to assume that the state of a target fish population can be known through random sampling using scientific vessels with the support of fishing industry vessels, joining efforts to collect information on size frequency, weights and reproductive stages of the individuals collected, while obtaining information on the physical/chemical environment where they were caught. This provides us with a correlation between the state of a population and its environment. This approach is quite useful and applied in many ecosystems, however, we still cannot understand the mechanism that results in the variability of the population in space/time, let alone its availability and future behavior for fishing.\\

Currently, we know that the early life stages in many pelagic fish species are critical to survive and then reach the recruitment goal, showing that mortality during these stages is highly dependent on the density and size of individuals \citep{StigRoge2019}. However, intraspecific variability has shown us that rapid growth increases the chances of survival and recruitment \citep{OsseBoog1997,Soga1997,MeekVigl2006}, accompanied by a scenario in which favorable environmental conditions are combined, such as moderate turbulence, which facilitates the search and capture of food, which should be abundant and in synchrony with the spawning process, ensuring rapid growth and therefore greater survival of the early life stages of a species \citep{CuryRoy1989}.\\

In the framework of this need for knowledge to understand the survival mechanism of the first early life stages of a target species, we have developed a tool that allows us to place individuals within a virtual marine environment to simulate processes of larval drift, growth, starvation and natural mortality. In this simulated environment, individuals are influenced by the variation in ocean currents generating different retention patterns off Peru; their metabolic rates will be affected by temperature variation and their energy supply will depend on the available food, showing different growth patterns. In this way, we ensure that we have a mechanism that connects the environment with each individual independently, explaining survival through the combined effect of ocean currents, temperature and food availability.\\

Thus, this research lays the methodological and scientific foundations by developing a model of egg and larval drift of Peruvian anchovy (\textit{Engraulis ringens}) accompanied by a first Dynamic Energy Budget (DEB) model of the complete life cycle for this species. We provide new insights into the understanding of \textit{E. ringens} recruitment, here understood as a dynamic and complex process
over different (eggs and larvae) life cycle stages \citep{DuffBail2005}. With this general objective in mind, this thesis is organized in 5 chapters: 1) to provide an introduction with the theoretical framework of the state of the art of the ecosystem and the species; 2) to explore the coastal retention process using a lagrangian modeling tool, which gives us an understanding of the potential presence of eggs and larvae in the continental shelf off Peru, catalogued as favorable to support high biomasses of various species; 3) developing a tool that combines the effect of temperature and food availability on growth and recruitment of \textit{E. ringens}, presenting a mechanism that will allow the environment to exert a direct impact on the individuals; 4) to develop a specific full life cycle model of \textit{E. ringens} and 5) to present a general overview of the research and to provide insights into future perspectives.\\
